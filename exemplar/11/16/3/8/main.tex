\def\inputGnumericTable{}
\let\negmedspace\undefined
\let\negthickspace\undefined
\documentclass[journal,12pt,twocolumn]{IEEEtran}
\usepackage{cite}
\usepackage{amsmath,amssymb,amsfonts,amsthm}
\usepackage{algorithmic}
\usepackage{graphicx}
\usepackage{textcomp}
\usepackage{xcolor}
\usepackage{txfonts}
\usepackage{listings}
\usepackage{enumitem}
\usepackage{mathtools}
\usepackage{gensymb}
\usepackage[breaklinks=true]{hyperref}
\usepackage{tkz-euclide} % loads  TikZ and tkz-base
\usepackage{listings}
\usepackage[latin1]{inputenc}
       \usepackage{fullpage}
       \usepackage{color}
       \usepackage{array}
       \usepackage{longtable}
       \usepackage{calc}
       \usepackage{multirow}
       \usepackage{hhline}
       \usepackage{ifthen}
       \usepackage{multirow}
\usepackage{adjustbox}



\DeclareMathOperator*{\Res}{Res}
\renewcommand\thesection{\arabic{section}}
\renewcommand\thesubsection{\thesection.\arabic{subsection}}
\renewcommand\thesubsubsection{\thesubsection.\arabic{subsubsection}}

\renewcommand\thesectiondis{\arabic{section}}
\renewcommand\thesubsectiondis{\thesectiondis.\arabic{subsection}}
\renewcommand\thesubsubsectiondis{\thesubsectiondis.\arabic{subsubsection}}                                

\lstset{
frame=single, 
breaklines=true,
columns=fullflexible
}

\begin{document}

\newtheorem{theorem}{Theorem}[section]
\newtheorem{problem}{Problem}
\newtheorem{proposition}{Proposition}[section]
\newtheorem{lemma}{Lemma}[section]
\newtheorem{corollary}[theorem]{Corollary}
\newtheorem{example}{Example}[section]
\newtheorem{definition}[problem]{Definition}
\newcommand{\BEQA}{\begin{eqnarray}}
\newcommand{\EEQA}{\end{eqnarray}}
\newcommand{\define}{\stackrel{\triangle}{=}}

\bibliographystyle{IEEEtran}

\providecommand{\mbf}{\mathbf}
\providecommand{\pr}[1]{\ensuremath{\Pr\left(#1\right)}}
\providecommand{\qfunc}[1]{\ensuremath{Q\left(#1\right)}}
\providecommand{\sbrak}[1]{\ensuremath{{}\left[#1\right]}}
\providecommand{\lsbrak}[1]{\ensuremath{{}\left[#1\right.}}
\providecommand{\rsbrak}[1]{\ensuremath{{}\left.#1\right]}}
\providecommand{\brak}[1]{\ensuremath{\left(#1\right)}}
\providecommand{\lbrak}[1]{\ensuremath{\left(#1\right.}}
\providecommand{\rbrak}[1]{\ensuremath{\left.#1\right)}}
\providecommand{\cbrak}[1]{\ensuremath{\left\{#1\right\}}}
\providecommand{\lcbrak}[1]{\ensuremath{\left\{#1\right.}}
\providecommand{\rcbrak}[1]{\ensuremath{\left.#1\right\}}}
\theoremstyle{remark}
\newtheorem{rem}{Remark}
\newcommand{\sgn}{\mathop{\mathrm{sgn}}}

\newcommand{\solution}{ \textbf{Solution: }}
\newcommand{\cosec}{\,\text{cosec}\,}
\providecommand{\dec}[2]{\ensuremath{\overset{#1}{\underset{#2}{\gtrless}}}}
\newcommand{\myvec}[1]{\ensuremath{\begin{pmatrix}#1\end{pmatrix}}}
\newcommand{\mydet}[1]{\ensuremath{\begin{vmatrix}#1\end{vmatrix}}}

\let\vec\mathbf


\vspace{3cm}

\title{
%	\logo{
Assignment 2
%	}
}
\author{ Gitanshu Arora % <-this % stops a space
}
% make the title area
\maketitle
\newpage
\bigskip
\renewcommand{\thefigure}{\theenumi}
\renewcommand{\thetable}{\theenumi}

\subsection*{\textbf{\underline{Problem 11.16.3.8(exemplar)}:-}}

A die is loaded in such a way that each odd number is twice as likely to occur as
each even number. Find P(G), where G is the event that a number greater than
3 occurs on a single roll of the die.

\subsection*{\textbf{\underline{Solution}:-}}

Let $X$ be a random variable denoting the number 
obtained on the die.\\

Let $m$ be any natural number such that $m\in\{1,2,3\}$.\\ 
% Let \pr{X=2m} be $p$,\\
% $\implies\pr{X=2m-1}=2p,\ where\ 1\leq m \leq3$
\begin{align}
\pr{X=k}= \begin{cases} 
      2p, & if\ k=2m-1 \\
      p, & if\ k=2m 
   \end{cases}
\end{align}
Let $F_X (k)$ be the cumulative distribution function such that,\\
\begin{align}
    F_X (k) = \pr{X\leq k}
\end{align}
If $k = 2m-1$,
\begin{align}
    F_X (k) = &\sum_{i=1}^{2m-1}{\pr{X=i}}
\\
    &= \sum_{i=1}^{m}{\pr{2i-1}} + \sum_{i=1}^{m-1}{\pr{2i}}\\
    &= \sum_{i=1}^{m}{2p} + \sum_{i=1}^{m-1}{p}\\
    &= (m)(2p) + (m-1)(p)\\
    &= p(3m-1)\\
    &= \frac{p(3k+1)}{2}
\end{align}
If $k = 2m$,
\begin{align}
    F_X (k) = &\sum_{i=1}^{2m-1}{\pr{X=i}}
\\
    &= \sum_{i=1}^{m}{\pr{2i-1}} + \sum_{i=1}^{m}{\pr{2i}}\\
    &=\sum_{i=1}^{m}{2p} + \sum_{i=1}^{m}{p}\\
    &= (m)(2p) + (m)(p)\\
    &= p(3m)\\
    &= \frac{3pk}{2}
\end{align}
So,
\begin{align}
&F_X (k)= \begin{cases} 
      \frac{p(3k+1)}{2}, & if\ k=2m-1 \\
      \frac{3pk}{2}, & if\ k=2m 
   \end{cases}
\end{align}\\
Since $1\leq X \leq6$,
\begin{align}
&F_X (6) = 1\\
&\implies \frac{3p(6)}{2} = 1\\
&\implies p = \frac{1}{9}
\end{align}
\begin{align}
  \pr{G} &= \pr{X>3}\\
  &=F_X (6) - F_X (3)\\
  &=\frac{3p(6)}{2} - \frac{p\{3(3)+1\}}{2}\\
  &= 9p - 5p = 4p\\
  &= \frac{4}{9}
  \end{align}
\end{document}



