\iffalse
\documentclass[12pt,onecolumn,notitlepage]{article}
\usepackage[margin=0.5in]{geometry}
\usepackage{amsmath}
\usepackage{gensymb}
\usepackage{graphicx}
\usepackage{amsthm}
\usepackage{mathrsfs}
\usepackage{txfonts}
\usepackage{cite}
\usepackage{cases}
\usepackage{subfig}
\usepackage[breaklinks=true]{hyperref}
\usepackage{listings}
\usepackage[latin1]{inputenc}
\usepackage{color}
\usepackage{array}
\usepackage{longtable}
\usepackage{calc}
\usepackage{multirow}
\usepackage{hhline}
\usepackage{ifthen}
\usepackage{amssymb}
\providecommand{\pr}[1]{\ensuremath{\Pr\left(#1\right)}}
\providecommand{\sbrak}[1]{\ensuremath{{}\left[#1\right]}}
\providecommand{\lsbrak}[1]{\ensuremath{{}\left[#1\right.}}
\providecommand{\rsbrak}[1]{\ensuremath{{}\left.#1\right]}}
\providecommand{\brak}[1]{\ensuremath{\left(#1\right)}}
\providecommand{\lbrak}[1]{\ensuremath{\left(#1\right.}}
\providecommand{\rbrak}[1]{\ensuremath{\left.#1\right)}}
\providecommand{\cbrak}[1]{\ensuremath{\left\{#1\right\}}}
\providecommand{\lcbrak}[1]{\ensuremath{\left\{#1\right.}}
\providecommand{\rcbrak}[1]{\ensuremath{\left.#1\right\}}}

\newcommand*{\comb}[2]{{}^{#1}C_{#2}}

\title{Probability Assignment 2 (12.13.3.3)}
\author{Aditya Varun V (AI22BTECH11001)}
\date{}

\begin{document}

\maketitle
\subsection*{Question}


\subsection*{Solution}
\fi

Let 
\begin{align}
	X &=  \begin{cases}
     0, \text{ if student is resides in hostel}\\
     1, \text{ if student is a day scholar} 
 \end{cases}\\
	Y &=  \begin{cases}
     0, \text{ if student does not attain A grade}\\
     1, \text{ if student attains A grade} 
 \end{cases}
\end{align}
From the given data, 
\begin{align}
	\pr{X=0} &= \frac{3}{5}\\
	\pr{X=1} &= \frac{2}{5}\\
	\pr{Y=1 \mid X=0} &= \frac{3}{10}\\
	\pr{Y=1 \mid X=1} &= \frac{1}{5}
\end{align}
 The desired probability is
 \begin{align}
	 \pr{X=0 \mid Y=1} &= \frac{\pr{Y=1 \mid X=0}\times\pr{X=0}}{\sum_{k=0}^{1}\pr{Y=1 \mid X=k}\times\pr{X=k} }\\
 	&= \frac{\frac{3}{10}\times\frac{3}{5}}{\frac{3}{10}\times\frac{3}{5} + \frac{1}{5}\times\frac{2}{5}}
 	= \frac{9}{13}
 \end{align}

