\iffalse
\documentclass[12pt,twocolumn]{article}
\usepackage[margin=0.5in]{geometry}
\usepackage[cmex10]{amsmath}
\usepackage{amsmath}
\usepackage{amsmath,amssymb,amsfonts}
\usepackage{graphicx}
\usepackage{textcomp}
\usepackage{amsmath,amssymb,amsfonts,amsthm}
\usepackage{multirow}
\usepackage{adjustbox}
\usepackage{gensymb}
\newcommand*{\Comb}[2]{{}^{#1}C_{#2}}%
\let\vec\mathbf
\title{
Probability Assignment
}
\author{GINNA SHREYANI}
\date{}
\providecommand{\pr}[1]{\ensuremath{\Pr\left(#1\right)}}


\begin{document}
\maketitle

\textbf{12.13.3.3}\\
Of the students in a college, it is known that 60\% reside in hostel and 40\% are day scholars (not residing in hostel). Previous year results report that 30\% of all students who reside in hostel attain A grade and 20\% of day scholars attain A grade in their annual examination. At the end of the year, one student is chosen at random from the college and he has an A grade, what is the probability that the student is a hostlier?
\subsection*{Solution}
Using Baye's Rule:\\
Let the probability of students living in hostel be $\pr{H=1}$, therefore the students who are day scholars can be given as $\pr{H=0}$\\
The probability of the students getting grade A is given as $\pr{A=1}$.\\

\begin{table}
	%%%%%%%%%%%%%%%%%%%%%%%%%%%%%%%%%%%%%%%%%%%%%%%%%%%%%%%%%%%%%%%%%%%%%%
%%                                                                  %%
%%  This is a LaTeX2e table fragment exported from Gnumeric.        %%
%%                                                                  %%
%%%%%%%%%%%%%%%%%%%%%%%%%%%%%%%%%%%%%%%%%%%%%%%%%%%%%%%%%%%%%%%%%%%%%%
\begin{align}
    \begin{tabular}{|l|l|}\hline
        Red ball from Bag I:	    &\pr{X=1,Y=1}=$\frac{3}{7}$\\\hline
        Black ball from Bag I:	&\pr{X=1,Y=2}=$\frac{4}{7}$\\\hline
        Red ball from Bag II:	&\pr{X=2,Y=1}=$\frac{31}{70}$\\\hline
        Black ball from Bag II: &\pr{X=2,Y=2}=$\frac{39}{70}$\\\hline
        \end{tabular}
    \end{align}
\end{table}

\begin{table}
        %%%%%%%%%%%%%%%%%%%%%%%%%%%%%%%%%%%%%%%%%%%%%%%%%%%%%%%%%%%%%%%%%%%%%%
%%                                                                  %%
%%  This is a LaTeX2e table fragment exported from Gnumeric.        %%
%%                                                                  %%
%%%%%%%%%%%%%%%%%%%%%%%%%%%%%%%%%%%%%%%%%%%%%%%%%%%%%%%%%%%%%%%%%%%%%%
\begin{tabular}{|l|l|l|l|l|}\hline
Results	&Profit	&Loss	&Total	&Probability\\\hline
4-H  0-T	&4	&0	&4	&$\frac{1}{16}$\\\hline
3-H  1-T	&3	&1.5	&1.5	&$\frac{1}{4}$\\\hline
2-H  2-T	&2	&3	&-1	&$\frac{3}{8}$\\\hline
1-H  3-T	&1	&4.5	&-3.5	&$\frac{1}{4}$\\\hline
0-H  4-T	&0	&6	&-6	&$\frac{1}{16}$\\\hline
\end{tabular}

\end{table}

Thus,
\begin{align}
	\pr{A=1} &= \sum_{i=0}^1 \pr{A = 1 \mid H = i}\pr{H=i}
\end{align}
\begin{multline}
	\pr{A=1} = \pr{A = 1 \mid H = 0}\pr{H = 0}\\
	+\pr{A = 1 \mid H = 1}\pr{H = 1}
\end{multline}
\begin{align}
	\pr{A=1} &= \left(\frac{20}{100} \times \frac{40}{100}\right)+\left(\frac{30}{100} \times \frac{60}{100}\right)\\
	\pr{A=1} &= \frac{26}{100}\\
	\pr{H=1 \mid A=1} &= \frac{\pr{A=1 \mid H=1}\pr{H=1}}{\pr{A=1}}\\
	\pr{H=1 \mid A=1} &= \frac{9}{13}
\end{align}
The probability that the student is a hostlier who has A grade is $\frac{9}{13}$.
\end{document}

